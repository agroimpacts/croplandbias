%% PNAStwoS.tex
%% Sample file to use for PNAS articles prepared in LaTeX
%% For two column PNAS articles
%% Version1: Apr 15, 2008
%% Version2: Oct 04, 2013

%% BASIC CLASS FILE
\documentclass{pnastwo}

%% ADDITIONAL OPTIONAL STYLE FILES Font specification

%\usepackage{pnastwoF}
\usepackage{cite}
\usepackage{lineno}
\setlength\linenumbersep{4pt}

%% OPTIONAL MACRO DEFINITIONS
\def\s{\sigma}
%%%%%%%%%%%%
%% For PNAS Only:
\url{www.pnas.org/cgi/doi/10.1073/pnas.0709640104}
\copyrightyear{2008}
\issuedate{Issue Date}
\volume{Volume}
\issuenumber{Issue Number}
%\setcounter{page}{2687} %Set page number here if desired
%%%%%%%%%%%%

\begin{document}

\title{Quantifying the impacts of bias in landcover data on global change analyses}

\author{Lyndon Estes\affil{1}{Princeton University, Princeton, NJ USA},
Peng Chen\affil{2}{Indiana University, Bloomington, IN USA},
Stephanie Debats\affil{1}{Princeton University, Princeton, NJ USA},
Tom Evans\affil{2}{Indiana University, Bloomington, IN USA},
Fanie Ferreira\affil{3}{GeoTerraImage, Pretoria, RSA},
Gabrielle Ragazzo\affil{1}{Princeton University, Princeton, NJ USA},
Justin Sheffield\affil{1}{Princeton University, Princeton, NJ USA}
\and
Kelly Caylor\affil{1}{Princeton University, Princeton, NJ USA}}

\contributor{Submitted to Proceedings of the National Academy of Sciences
of the United States of America}

%%%Newly updated.
%%% If significance statement need, then can use the below command otherwise just delete it.
%\significancetext{LDE developed the concept of the study, conducted the analysis, data interpretation and drafted the manuscript. \color{red}{To be completed}}

\maketitle

\begin{article}
\begin{abstract}
{Blah blah.}
\end{abstract}

\keywords{landcover | bias | remote sensing | agriculture | crop yield | harvested area | carbon | agent-based model | landscape}

\abbreviations{GTI, GeoTerraImage; SSA, sub-Saharan Africa}
\linenumbers

\dropcap{T}he nature and distribution of landcover is a fundamental determinant of many environmental and social processes that drive or are affected by global change \cite{lambin_modelling_1997}, such as agricultural production and food security \cite{lark_cropland_2015,wright_recent_2013, licker_mind_2010}, carbon cycling \cite{asner_high-resolution_2010,gaveau_major_2014}, biodiversity loss \cite{newbold_global_2015,luoto_predicting_2004}, or demographic changes \cite{linard_assessing_2010}. Landcover maps are therefore critical for understanding the nature and impact of such changes \cite{see_improved_2015}, and they need to be accurate at the finest scales at which the underlying processes operate. For example, agricultural productivity and nutrient loadings can vary greatly between neighboring fields, and field sizes are often $<$2 hectares in regions where smallholder farming still dominates \cite{jain_mapping_2013, debats_generalized_????}. To understand agriculturally driven processes, it is thus necessary to accurately delineate fields at their smallest grain size, and to do so at regional to global scales to have a consistent set of maps.   

Landcover data can only be developed with satellite imaging, but often the average size class of the cover type of interest is smaller than the sensor resolution, or spectrally indistinct from other neighboring covers, which propagates classification error \cite{see_improved_2015,lobell_use_2013,estes_diylandcover:_2015}. The result is that landcover datasets are generally inaccurate at finer scales and greatly differ between one another, particularly in those parts of the world undergoing the most rapid land use changes, where the aforementioned sources of bias tend to be most pronounced \cite{estes_projected_2013,fritz_comparison_2010,fritz_cropland_2011}.  

These errors are well-known \cite{fritz_comparison_2010, fritz_cropland_2011, see_improved_2015, fritz_mapping_2015,verburg_challenges_2011}, and there are a variety of efforts underway to improve landcover maps, particularly for agriculture \cite{fritz_geo-wiki:_2012,estes_diylandcover:_2015}. What is less known is the degree to which these errors bias measurements built upon the distributional and areal information in landcover. An impediment to this understanding is that the errors are hard to quantify because spatially extensive reference data are not available for most regions of the world--particularly over Africa and other developing regions. Errors assessment therefore typically rely on a small number of ground truth points or survey data aggregated to political boundaries. For this reason, we have a better understanding of the biases between landcover datasets or in relation to country-level statistics \cite{fritz_comparison_2010,fritz_cropland_2011,kaptue_tchuente_comparison_2011} than we do of how error changes over spatial gradients or as a function of aggregation scale. 

Being unable to fully quantify the errors in landcover maps of course makes it difficult, if not impossible, to quantify their impact on downstream analyses. There has been some work examining how such error influences climate simulations \cite{ge_impacts_2007}, agricultural land use patterns \cite{schmit_limitations_2006}, and carbon flux \cite{quaife_impact_2008} and human population estimates \cite{linard_assessing_2010}, but these either use simulated landcover errors \cite{ge_impacts_2007} or compare relevant differences in estimates between different satellite-derived landcover maps \cite{linard_assessing_2010, quaife_impact_2008}. The exception is \cite{schmit_limitations_2006}, who use a high quality, ground-collected reference map detailing farm land use parcels in central Belgium, but the number of sites and region were both fairly restricted, and the parcels were not spatially contiguous. 

There is thus an urgent need to more precisely quantify landcover map errors and how these vary over large regions, particularly for the regions where landcover is changing most rapidly yet is most poorly known.  We address this need in this study, using a unique, high accuracy agricultural landcover map for South Africa to quantify the errors in several latest generation landcover maps that are broadly used in global change studies.  We use this information to examine how i) landcover properties and related classification schemes influence error, ii) how these errors change with aggregation scale, with the specific goal of determining ``safe'' scales for making area-based calculations, and 3) how these errors propagate through several different forms of downstream analyses that broadly represent the global change research focus areas, including biogeochemical and land use change studies, food security assessments, land surface hydrology and climatology, and human geography.  

\vspace{-0.5 cm}
\section{Study area and landcover data}
Our study focused on South Africa, which comprises nearly 6\% of sub-Saharan Africa's (SSA) landmass, and has a large, diverse agricultural sector, ranging from large commercial operations to smallholder farms \cite{hardy_rainfed_2011,estes_using_2014}. This diversity suggests that the country's agricultural landcover spans the range of types that are found throughout the rest of SSA.  

The South African government commissioned a whole country cropland boundary map in order to stratifying the annual aerial crop type census used to calculate harvested area estimates \cite{fourie_better_2009}. The map was made by trained workers who visually interpreted high resolution satellite imagery and manually digitized field boundaries following a standardized mapping protocol. The resulting vectorized field maps, which were made in 2007 and updated in 2011, provide a unique, high accuracy reference dataset of both crop field distribution and size classes.  We converted the vector data into a rasterized estimated of cropland percentage at 1 km resolution (henceforth the ``reference map''), which was 97\% accurate in distinguishing cropped from non-cropped areas. 

We compared our reference percent cropland estimates to those created from four satellite-derived landcover datasets. We obtained South Africa's 30 m resolution National Landcover map (SA-LC) for 2009 \cite{sanbi_national_2009}, the 500 m resolution MODIS Landcover for 2011 \cite{land_processes_distributed_active_archive_center_lp_daac_modis_2011, friedl_modis_2010}, the 300 m resolution GlobCover 2009 \cite{arino_global_2012}, and the new 1 km Geo-wiki hybrid-fusion cropland map for Africa \cite{fritz_mapping_2015}. We chose these particular datasets because they are nearly contemporaneous with our reference data, and represent the major types of landcover products used by researchers: SA-LC typifies the higher resolution, Landsat-derived maps that are developed individually for many countries \cite{fry_completion_2009},  MODIS and GlobCover are widely used global-scale products \cite{gross_monitoring_2013,shackelford_conservation_2015}, while Geo-Wiki incorporates the first three datasets and is the current state of the art for agricultural landcover maps. We extracted the cropland classes from the first three datasets and converted these to 1 km resolution percent cropland estimates (hereafter simply ``cropland maps''), resulting in 4 maps to compare to the reference.  

\vspace{-0.5 cm}
\section{Quantifying Error}
We used these maps to first quantify error in cropland area estimates. We calculated error as the difference between the reference and cropland map at different scales of aggregation (1 to 100 km), to determine the extent of bias in each map and how it varies with scale. Next, we assessed how map error relates to cover patterns in agricultural  landscapes, by examining how map error over cropped pixels correlates with the actual amount of cropland.    

We undertook five further analyses to investigate how map error can impact assessments that are founded on landcover maps. These include first-order analyses, in which values for a variable of interest are mapped to  particular cover type(s), and second-order analyses, in which a process model draws on the cover types' values to calculate an output value. We created four datasets to represent second order analyses. The first was a series of maps of vegetated carbon stocks created following the methodology of Ruesch and Gibbs' \cite{ruesch_new_2008}. The second was cropland percentage maps, which, following Ramankutty et al \cite{ramankutty_farming_2008} were adjusted so that their total areas matched the reported provincial-level cropland area. Using these adjusted cropland percentage maps, we followed the methods of Monfreda et al \cite{monfreda_farming_2008} to disaggregate census-reported maize harvested area and yields. We then compared differences between total carbon stock estimates calculated from the reference map with those from the four cropland maps, and again examined how these differences changed as a function of aggregation scale.  We made the same comparisons for total maize harvested area, average yield, and total production. 

For the second-order analyses, we examined how cropland cover errors influence 25 km resolution monthly evapotranspiration estimates produced using the Variable Infiltration Capacity \cite{liang_simple_1994}. For this example, we used the cropland maps to adjust the seasonally varying, landcover-specific leaf area index (LAI) values that VIC uses to partition water vapor fluxes into their evaporative and transpirative components. In the second example, we examined how these errors can impact the parameterization of a computationally intensive, spatially explicit agent-based model of food security. In this case, we used the cropland maps to allocate farmland to model agents representing individual households in political districts, with each agent's initial holdings assigned as a function of the total cropland available in the district, and the proximity of cropland within a specified distance of agent's location.  


%\vspace{-0.5 cm}
\section{Results}

\subsection{Percent cropland estimates}

\subsection{Error as a function of cropland density}

\subsection{Potential bias in harvested areas, yield, and production estimates}

\subsection{Potential bias in estimates of carbon stocks}

\subsection{Potential bias in harvested areas, yield, and production estimates}

\subsection{Impacts on evapotranspiration estimates}

\subsection{Initialization errors in spatial agent-based models}

%large gradient ($\sim \displaystyle{ \inlinefrac{1}{\delta}}$).

\section{Discussion}
\subsection{Blather}
% $\theta$ changes from 0 to 1. (see Figure \ref{afoto}).That means we are considering $\theta$  of the form
%For these solutions we have the following

%\begin{remark}
%Note that equation \eqref{theta} specifies  the function $\varphi$
%up to an error of order $\delta$. Theorem 1 provides an evolution
%equation for the function $\varphi$ up to an error of order
%$\delta |log \delta|$.
%\end{remark}

%(see \eqref{weaksol}). 
\subsection{More blather}


\begin{materials}
\section{Methods} 
Perhaps it is right {\it SI Materials and Methods}.

\section{Digital RCD Analysis} 

\end{materials}

\appendix[App 1]

\appendix
This is an example of an appendix without a title.

\begin{acknowledgments}
I thank everyone tearfully. 
\end{acknowledgments}


% bib solution from here
% http://tex.stackexchange.com/questions/167650/is-there-a-more-recent-bibliography-style-file-bst-for-pnas
% https://github.com/jburon/pnas2011.bst
\bibliographystyle{pnas2011} 
\bibliography{bias}

%\begin{thebibliography}{10}
%\bibitem{BN}
%M.~Belkin and P.~Niyogi, {\em Using manifold structure for partially
%  labelled classification}, Advances in NIPS, 15 (2003).
%
%\bibitem{BBG:EmbeddingRiemannianManifoldHeatKernel}
%P.~B\'erard, G.~Besson, and S.~Gallot, {\em Embedding {R}iemannian
%  manifolds by their heat kernel}, Geom. and Fun. Anal., 4 (1994),
%  pp.~374--398.
%\end{thebibliography}


\end{article}

%\begin{figure}
%\centerline{\includegraphics[width=.4\textwidth]{figsamp.eps}}
%\caption{LKB1 phosphorylates Thr-172 of AMPK$\alpha$ \textit{in vitro}
%and activates its kinase activity.}\label{afoto}
%\end{figure}
%
%\begin{figure*}[ht]
%\begin{center}
%\centerline{\includegraphics[width=.7\textwidth]{figsamp.eps}}
%\caption{LKB1 phosphorylates Thr-172 of AMPK$\alpha$ \textit{in vitro}
%and activates its kinase activity.}\label{afoto2}
%\end{center}
%\end{figure*}
%
%\begin{table}[h]
%\caption{Repeat length of longer allele by age of onset class.
%This is what happens when the text continues.}
%\begin{tabular}{@{\vrule height 10.5pt depth4pt  width0pt}lrcccc}
%&\multicolumn5c{Repeat length}\\
%\noalign{\vskip-11pt}
%Age of onset,\\
%\cline{2-6}
%\vrule depth 6pt width 0pt years&\multicolumn1c{\it n}&Mean&SD&Range&Median\\
%\hline
%Juvenile, 2$-$20&40&60.15& 9.32&43$-$86&60\\
%Typical, 21$-$50&377&45.72&2.97&40$-$58&45\\
%Late, $>$50&26&41.85&1.56&40$-$45&42\tablenote{The no. of wells for all samples was 384. Genotypes were
%determined by mass spectrometric assay. The $m_t$ value indicates the
%average number of wells positive for the over represented allele.}
%\\
%\hline
%\end{tabular}
%\end{table}
%
%
%\begin{table*}[ht]
%\caption{Summary of the experimental results}
%\begin{tabular*}{\hsize}
%{@{\extracolsep{\fill}}rrrrrrrrrrrrr}
%\multicolumn{3}{l}{Parameters}&
%\multicolumn{5}{c}{Averaged Results}&
%\multicolumn{5}{c}{Comparisons}\cr
%\hline
%\multicolumn1c{$n$}&\multicolumn1c{$S^*_{MAX}$}&
%\multicolumn1c{$t_1$}&\multicolumn1c{\ $r_1$}&
%\multicolumn1c{\ $m_1$}&\multicolumn1c{$t_2$}&
%\multicolumn1c{$r_2$}&\multicolumn1c{$m_2$}
%&\multicolumn1c{$t_{lb}$}&\multicolumn1c{\ \ $t_1/t_2$}&
%$r_1/r_2$&$m_1/m_2$&
%$t_1/t_{lb}$\cr
%\hline
%10\tablenote{Stanford Synchrotron Radiation Laboratory (Stanford University,
%Stanford, CA)}&1\quad &4&.0007&4&4&.0020&4&4&1.000&.333&1.000&1.000\cr
%10\tablenote{$R_{\rm FREE}=R$ factor for the $\sim 5$\% of the randomly
%chosen unique ref\/lections not used in the ref\/inement.}&5\quad &50&.0008&8&50&.0020&12&49&.999&.417&.698&1.020\cr
%100\tablenote{Calculated for all observed data}&20\quad &2840975&.0423&95&2871117&.1083&521&---&
%.990&.390&.182&---\ \ \cr
%\hline
%\end{tabular*}
%\end{table*}


\end{document}


