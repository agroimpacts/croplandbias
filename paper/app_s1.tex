\RequirePackage{lineno}  % http://www-d0.fnal.gov/Run2Physics/WWW/templates/lineno.html
%\documentclass[12pt]{iopart}
\documentclass[11pt, titlepage]{article}
\usepackage[top=20mm, bottom=20mm, left=23mm, right=23mm]{geometry}
%\newcommand{\gguide}{{\it Preparing graphics for IOP journals}}
%Uncomment next line if AMS fonts required
%\usepackage{iopams}  
% parameters from other files
\newcommand{\citetapos}[1]{\citeauthor{#1}'s \citeyearpar{#1}}  % apostrophe's in cites
%\usepackage[square,numbers,sort&compress]{natbib}
\def\linenumberfont{\normalfont\small\sffamily}
\usepackage{graphicx}
\usepackage{epstopdf}
% From link below: next two lines allow caption to fill the full figure box, and caption formats to be changed (e.g. font)
% http://tex.stackexchange.com/questions/107350/caption-below-the-figure-and-aligned-with-left-side-of-figure
% http://ctan.mackichan.com/macros/latex/contrib/caption/caption-eng.pdf
\usepackage{caption}
\captionsetup[figure]{slc=off, font=footnotesize}
%\usepackage{amsmath}
\usepackage{longtable}
\usepackage{placeins}
\usepackage{multirow}
\usepackage[numbers,round,sort,compress]{natbib}
\bibpunct{(}{)}{,}{n}{,}{,}  % https://xianblog.wordpress.com/tag/natbib/ (allows natbib with PNAS)


\begin{document}
%\bibliographystyle{unsrt}

\setpagewiselinenumbers
%\modulolinenumbers[5]
%\linenumbers
\section*{\large Appendix S1: Supplementary Results}
%\begin{table}[ht]
%\centering
%\caption{Confidence classes for the forcing data based on bias-correction weights.}
%\begin{tabular}{cc}
%  \hline
%  Confidence class & Bias-correction weight \\ 
%  \hline
%High & 0.75-1.00 \\ 
%Medium-High & 0.50-0.75\\ 
%Medium-Low & 0.25-0.50\\ 
% Low & 0.05-0.25 \\ 
% None & $<$0.05 \\ 
%   \hline
%\end{tabular}
%\end{table}  
\section*{Reference data accuracy}

The accuracy of the reference vector dataset was assessed by Estes et al \cite{estes_platform_2015}. The assessment was undertaken using a sub-sample of 609 1 km$^2$ grid cells, which were selected using a weighted randomized sampling scheme. Weights were derived from a logistic regression model of cropland occurrence probability in South Africa, with 1 corresponding to the lowest quartile of probability, and 4 the highest. The accuracy assessment included all cropland classes mapped within the reference dataset, which ranged from communal/smallholder fields to commercial row crops to orchards and other types of horticulture. An visual accuracy assessment was conducted within each 1 km$^2$ grid cell, wherein each cell was divided into 25 smaller cells of 200 X 200 m (4 ha), and then the proportion of each cell filled with cropfields visible in underlying Google Maps imagery was then calculated to the nearest 5\% of coverage, using a finer 20-cell mesh overlaid on each sub-cell within which field presence/absence was recorded. The same procedure was then performed to assess coverage by reference map polygons, as well as the intersections and differences between cells determined to be occupied by visual assessment, and those by the reference polygons. These intersections and differences were used to calculate the area of true positives, false positives, true negatives, and false negatives within each grid cell, and from this obtain accuracy measures.  In this case, accuracy was assessed using two landcover classes: cropland and non-cropland, thus a two class confusion matrix was constructed (Table 1) according to remote sensing classification accuracy guidelines set out by Oloffson et al. \cite{olofsson_good_2014}. In this case, the reference data were the visually interpreted crop field presences/absences, and the map data were the field boundary vectors being assessed (i.e. the data which provide the reference maps in this study), the class ``crop" refers to crop field presence and ``non-crop" to crop field absence, or all other landcover types.  The total area of vectorized crop fields was used to calculate the proportion \emph{W} of South Africa's total area mapped as crop fields, and of the areas ``mapped'' (by omission) as having other cover types.  These were then used to weight the proportions of each mapped class \emph{i} corresponding to reference class \emph{j} by the total area \emph{A} mapped for class \emph{i}. The total accuracy (0.97, or 97\%) was calculated by summing the diagonal (bold in Table 1), as well as the producer's accuracy \emph{P} for each reference class \emph{j}, and the user's accuracy \emph{U} for each mapped class \emph{i}.  

\begin{table}[ht]
\centering
\caption{Confusion matrix for the assessment of the hand-digitized crop field boundaries used to generate reference cropland cover percentage values. Here \emph{Reference} denotes a visual assessment of crop field presence in high resolution imagery within a sample of 609 sites, while \emph{Map} refers to the vectorized crop field boundaries. The assessment had two classes: crop fields and non-crop fields, and proportions corresponding to these two categories are the proportions of areas determined by class agreements and disagreement in the 609 sites (see text above), which are weighted by the proportion \emph{W} of each class' total mapped area \emph{A$_i$} in South Africa. \emph{P} and \emph{U} provide the Producer's and User's accuracies, respectively. }
\begin{tabular}{rrrrrrr}
  \hline
  & & Reference (j) & & & & \\\cline{3-7}
  & & Crop & Non-crop & \emph{W}$_i$ & \emph{A}$_i$ (ha) & \emph{U}$_i$ \\ 
  \hline
  Map (i) & Crop & \textbf{0.108} & 0.007 & 0.114 & 14018567 & 0.943 \\ 
  & Non-crop & 0.021 & \textbf{0.865} & 0.886 & 108541733 & 0.977 \\ 
  & Total & 0.128 & 0.872 & 1.000 & 122560300 &  \\ 
  & \emph{P}$_j$ & 0.840 & 0.992 &  &  &  \\ 
   \hline
\end{tabular}
\end{table}

\section*{Landcover map bias}

% latex table generated in R 3.4.1 by xtable 1.8-2 package
% Sun Aug 27 16:09:11 2017
\begin{longtable}{lllrrrrrr}
\caption{Biases and mean absolute errors (MAE) in cropland maps relative to the 2011 reference map for each aggregation scale calculated over the entire country, for the union of agricultural regions (cropland $>$ 0), and as density independent means, wherein the mean bias/MAE values for each of 20 cropland cover classes (representing 5\% increments of cover 0\% to 100\% defined by the reference map) were calculated and then averaged.} \\ 
  \hline
Region & Metric & Map & 1 km & 5 km & 10 km & 25 km & 50 km & 100 km \\ 
  \hline
Country & Bias & SA-LC & -2.5 & -2.5 & -2.5 & -2.5 & -2.5 & -2.5 \\ 
  Country & Bias & GlobCover & 2.0 & 2.0 & 2.0 & 2.0 & 2.0 & 2.0 \\ 
  Country & Bias & MODIS & 2.5 & 2.5 & 2.5 & 2.5 & 2.5 & 2.5 \\ 
  Country & Bias & GLC-Share & -0.6 & -0.6 & -0.6 & -0.6 & -0.6 & -0.6 \\ 
  Country & MAE & SA-LC & 3.3 & 2.9 & 2.8 & 2.7 & 2.6 & 2.6 \\ 
  Country & MAE & GlobCover & 11.3 & 9.4 & 8.8 & 8.2 & 7.6 & 7.2 \\ 
  Country & MAE & MODIS & 7.8 & 6.5 & 6.1 & 5.6 & 5.0 & 4.6 \\ 
  Country & MAE & GLC-Share & 6.3 & 4.5 & 3.9 & 3.2 & 2.6 & 2.2 \\ 
  Agricultural & Bias & SA-LC & -9.1 & -4.3 & -3.3 & -2.9 & -2.7 & -2.6 \\ 
  Agricultural & Bias & GlobCover & 3.7 & 2.8 & 2.5 & 2.2 & 2.1 & 2.1 \\ 
  Agricultural & Bias & MODIS & 7.1 & 4.1 & 3.4 & 2.9 & 2.7 & 2.6 \\ 
  Agricultural & Bias & GLC-Share & -2.0 & -1.0 & -0.8 & -0.6 & -0.6 & -0.6 \\ 
  Agricultural & MAE & SA-LC & 11.1 & 5.0 & 3.7 & 3.0 & 2.8 & 2.6 \\ 
  Agricultural & MAE & GlobCover & 20.5 & 12.8 & 10.7 & 9.0 & 8.1 & 7.3 \\ 
  Agricultural & MAE & MODIS & 22.7 & 10.8 & 8.2 & 6.3 & 5.4 & 4.7 \\ 
  Agricultural & MAE & GLC-Share & 19.2 & 7.8 & 5.3 & 3.7 & 2.9 & 2.2 \\ 
  Density independent & Bias & SA-LC & -8.6 & -7.6 & -7.7 & -7.0 & -6.5 & -5.5 \\ 
  Density independent & Bias & GlobCover & 33.6 & 33.5 & 30.9 & 27.9 & 23.8 & 14.4 \\ 
  Density independent & Bias & MODIS & 20.9 & 13.9 & 11.0 & 8.1 & 7.7 & 7.1 \\ 
  Density independent & Bias & GLC-Share & 4.3 & -1.1 & -1.8 & -2.3 & -0.1 & -0.6 \\ 
  Density independent & MAE & SA-LC & 11.4 & 8.4 & 8.1 & 7.2 & 6.6 & 5.6 \\ 
  Density independent & MAE & GlobCover & 37.1 & 35.8 & 32.9 & 29.7 & 25.9 & 16.9 \\ 
  Density independent & MAE & MODIS & 31.5 & 22.3 & 19.4 & 16.9 & 14.7 & 11.2 \\ 
  Density independent & MAE & GLC-Share & 23.2 & 12.1 & 9.9 & 8.1 & 6.2 & 3.8 \\ 
   \hline
\hline
\end{longtable}


\begin{figure}[ht]
  \centering
     \includegraphics[width = 16cm]{figures/cropland_bins.pdf} 
%     \vspace{-3cm}
      \caption{Number of cells within each cropland density bin at each scale of aggregation, where bins represent 5\% increment of cropland cover from 0 to 100\% (values on x-axis provide the upper limit of each bin).}
      \label{fig:default}
\end{figure}


% latex table generated in R 3.4.1 by xtable 1.8-2 package
% Sun Aug 27 16:11:21 2017
\begin{longtable}{lllrrrrrr}
\caption{Biases and mean absolute errors (MAE) in cropland maps relative to the 2007 reference map for each aggregation scale calculated over the entire country, for the union of agricultural regions, and as density independent means, wherein the mean bias/MAE values for each of 20 cropland cover classes (representing 5\% increments of cover 0\% to 100\% defined by the reference map) were calculated and then averaged.} \\ 
  \hline
Region & Metric & Map & 1 km & 5 km & 10 km & 25 km & 50 km & 100 km \\ 
  \hline
Country & Bias & SA-LC & -2.9 & -2.9 & -2.9 & -2.9 & -2.9 & -2.9 \\ 
  Country & Bias & GlobCover & 1.7 & 1.7 & 1.7 & 1.7 & 1.7 & 1.7 \\ 
  Country & Bias & MODIS & 2.2 & 2.2 & 2.2 & 2.2 & 2.2 & 2.2 \\ 
  Country & Bias & GLC-Share & -0.9 & -0.9 & -0.9 & -0.9 & -0.9 & -0.9 \\ 
  Country & MAE & SA-LC & 3.0 & 2.9 & 2.9 & 2.9 & 2.9 & 2.9 \\ 
  Country & MAE & GlobCover & 11.3 & 9.4 & 8.9 & 8.2 & 7.7 & 7.2 \\ 
  Country & MAE & MODIS & 7.7 & 6.4 & 6.0 & 5.5 & 5.0 & 4.6 \\ 
  Country & MAE & GLC-Share & 6.2 & 4.4 & 3.8 & 3.1 & 2.6 & 2.2 \\ 
  Agricultural & Bias & SA-LC & -9.9 & -4.9 & -3.8 & -3.2 & -3.1 & -2.9 \\ 
  Agricultural & Bias & GlobCover & 3.1 & 2.3 & 2.0 & 1.9 & 1.8 & 1.7 \\ 
  Agricultural & Bias & MODIS & 6.4 & 3.6 & 2.9 & 2.5 & 2.3 & 2.2 \\ 
  Agricultural & Bias & GLC-Share & -2.8 & -1.6 & -1.2 & -1.0 & -1.0 & -0.9 \\ 
  Agricultural & MAE & SA-LC & 10.4 & 5.0 & 3.9 & 3.3 & 3.1 & 3.0 \\ 
  Agricultural & MAE & GlobCover & 20.6 & 12.8 & 10.7 & 9.1 & 8.1 & 7.4 \\ 
  Agricultural & MAE & MODIS & 22.6 & 10.6 & 8.0 & 6.2 & 5.3 & 4.7 \\ 
  Agricultural & MAE & GLC-Share & 19.2 & 7.7 & 5.2 & 3.6 & 2.8 & 2.2 \\ 
  Density independent & Bias & SA-LC & -9.9 & -8.5 & -8.4 & -7.4 & -6.9 & -6.6 \\ 
  Density independent & Bias & GlobCover & 33.8 & 33.8 & 31.2 & 27.4 & 21.6 & 14.8 \\ 
  Density independent & Bias & MODIS & 20.8 & 13.3 & 10.5 & 7.0 & 6.9 & 7.0 \\ 
  Density independent & Bias & GLC-Share & 3.8 & -1.8 & -2.5 & -2.8 & -0.8 & -1.5 \\ 
  Density independent & MAE & SA-LC & 10.5 & 8.6 & 8.4 & 7.4 & 6.9 & 6.6 \\ 
  Density independent & MAE & GlobCover & 37.4 & 36.1 & 33.2 & 29.3 & 24.0 & 17.4 \\ 
  Density independent & MAE & MODIS & 31.7 & 22.2 & 19.5 & 17.2 & 14.5 & 11.6 \\ 
  Density independent & MAE & GLC-Share & 23.3 & 12.0 & 9.8 & 8.2 & 5.9 & 3.9 \\ 
   \hline
\hline
\end{longtable}



\begin{figure}[ht]
  \centering
     \includegraphics[width = 16cm]{figures/biases_1-100km.pdf} 
%     \vspace{-3cm}
      \caption{Biases and mean absolute errors (MAE) for each of the cropland maps as a function of cropland density (calculated using the 2011 reference maps) and aggregation scales. Rows present biases by map product, columns by aggregation scale.  Dash lines indicate bias at each level of cropland density, calculated in bins spanning 5\% of density (e.g. 0-5\% cropland cover, 5-10\%, etc.), while solid lines indicate the mean absolute error.  The black numbers in each plot area present the overall means of bias/MAE for each sensor-scale combination. The bin-wise and overall mean statistics were calculated from pooled map errors calculated from differences between the 2007 reference map and each cropland map (including all three variants--high, medium, and low--of the MODIS and GlobCover-derived cropland maps), and the 2011 reference map and each cropland map. }
      \label{fig:default}
\end{figure}

%\vspace{0.5 cm}
%\begin{equation}
%\hspace*{1.5 cm} 
%   \label{metric}
%     R_n  = \frac{(1 - \alpha)dswr + dlwr - \sigma * T^4}{\lambda}
%\end{equation}

\begin{figure}[ht]
  \centering
     \includegraphics[width = 12cm]{figures/biases_1km.pdf} 
%     \vspace{-3cm}
      \caption{Biases and mean absolute errors (MAE) for each of the cropland maps at 1 km resolution, as a function of cropland density. Colored lines (color-coded to map product name) show the bias/MAE at each level of cropland density, calculated in bins spanning 5\% (e.g. 0-5\% cropland cover, 5-10\%, etc.). Box plots show the variability of bias in each bin (whiskers = 2.5 and 97.5 percentiles, box the inter-quartile, and grey bar in box the median). Biases are presented in the top row, and MAEs in the bottom row. Statistics were calculated from pooled map errors calculated from differences between the 2007 reference map and each cropland map (including all three variants--high, medium, and low--of the MODIS and GlobCover-derived cropland maps), and the 2011 reference map and each cropland map. }
      \label{fig:default}
\end{figure}

\begin{figure}[ht]
  \centering
     \includegraphics[width = 8cm]{figures/md_map.pdf} 
%     \vspace{-3cm}
      \caption{South Africa's magisterial districts.}
      \label{fig:default}
\end{figure}

\FloatBarrier
\section*{Carbon bias}
\begin{figure}[ht]
  \centering
     \includegraphics[width = 12cm]{figures/carbon_bias_map.pdf} 
%     \vspace{-3cm}
      \caption{Spatial patterns of error (averaged across four different possible cover types adjacent to cropland) in carbon stock estimates. }
      \label{fig:default}
\end{figure}

% latex table generated in R 3.4.1 by xtable 1.8-2 package
% Mon Aug 28 06:26:20 2017
\begin{table}[ht]
\centering
\caption{Percent differences in total carbon stock estimates calculated from the reference maps and from each of the four cropland maps. Differences are evaluated for total carbon estimates either at the country scale or over just the agricultural regions (cropland $>$0.05\%), using the carbon densities of 5 different cover types to provide the values for the non-agricultural portions of each pixel (cover types indicated by column names).} 
\begin{tabular}{llrrrrr}
  \hline
Region & Map & Forest & Secondary & Shrubland & Grassland & Sparse \\ 
  \hline
Country & SA-LC & 2.6 & 2.5 & 2.5 & 0.1 & -2.1 \\ 
  Country & GlobCover & -2.1 & -2.0 & -2.0 & -0.1 & 1.7 \\ 
  Country & MODIS & -2.6 & -2.5 & -2.5 & -0.1 & 2.1 \\ 
  Country & GLC-Share & 0.6 & 0.5 & 0.5 & 0.0 & -0.5 \\ 
  Agricultural & SA-LC & -2.0 & -2.7 & -2.8 & -10.6 & -14.9 \\ 
  Agricultural & GlobCover & -161.9 & -156.3 & -155.5 & -95.9 & -63.6 \\ 
  Agricultural & MODIS & -1.6 & -0.8 & -0.7 & 8.4 & 13.3 \\ 
  Agricultural & GLC-Share & 7.7 & 7.3 & 7.2 & 2.9 & 0.5 \\ 
   \hline
\end{tabular}
\end{table}


\FloatBarrier
% latex table generated in R 3.2.1 by xtable 1.7-4 package
% Sat Oct 17 19:46:02 2015
\begin{longtable}{llllrrrrrr}
\caption{Biases and mean absolute errors for each of the cropland maps across aggregation scales and each possible landcover type sharing the pixel with cropland. Three variants of the bias/accuracy values are provided, indicated by the: Region column: 1) weighted by reference cropland density (Density weighted); 2) across the entire country (Country); 3) for agricultural areas only (Agricultural, i.e. $>$0.05\% cropland).} \\ 
  \hline
Region & Metric & Map & Cover & 1 km & 5 km & 10 km & 25 km & 50 km & 100 km \\ 
  \hline
Agricultural & Bias & GeoWiki & All & -5.6 & 0.8 & 0.8 & 0.5 & 0.4 & 0.4 \\ 
  Agricultural & Bias & GlobCover & All & -23.9 & -8.5 & -6.5 & -4.7 & -3.6 & -2.7 \\ 
  Agricultural & Bias & MODIS & All & -22.1 & -5.3 & -3.9 & -2.9 & -2.4 & -1.9 \\ 
  Agricultural & Bias & SA-LC & All & 7.2 & 4.0 & 3.2 & 2.6 & 2.2 & 1.9 \\ 
  Agricultural & Bias & GeoWiki & Forest & -11.9 & 1.9 & 1.7 & 1.2 & 0.9 & 0.8 \\ 
  Agricultural & Bias & GlobCover & Forest & -51.9 & -16.6 & -12.5 & -9.0 & -6.9 & -5.3 \\ 
  Agricultural & Bias & MODIS & Forest & -47.8 & -10.9 & -7.9 & -6.0 & -5.1 & -4.2 \\ 
  Agricultural & Bias & SA-LC & Forest & 15.6 & 8.6 & 7.0 & 5.5 & 4.8 & 4.2 \\ 
  Agricultural & Bias & GeoWiki & Secondary & -6.9 & 1.6 & 1.5 & 1.1 & 0.8 & 0.8 \\ 
  Agricultural & Bias & GlobCover & Secondary & -34.2 & -13.2 & -10.3 & -7.8 & -6.1 & -4.7 \\ 
  Agricultural & Bias & MODIS & Secondary & -33.3 & -9.4 & -7.0 & -5.5 & -4.7 & -3.9 \\ 
  Agricultural & Bias & SA-LC & Secondary & 13.5 & 7.6 & 6.2 & 5.0 & 4.4 & 3.9 \\ 
  Agricultural & Bias & GeoWiki & Shrubland & -6.5 & 1.6 & 1.4 & 1.1 & 0.8 & 0.8 \\ 
  Agricultural & Bias & GlobCover & Shrubland & -32.6 & -12.8 & -10.0 & -7.6 & -6.0 & -4.7 \\ 
  Agricultural & Bias & MODIS & Shrubland & -31.9 & -9.2 & -6.9 & -5.4 & -4.7 & -3.9 \\ 
  Agricultural & Bias & SA-LC & Shrubland & 13.2 & 7.5 & 6.1 & 5.0 & 4.4 & 3.8 \\ 
  Agricultural & Bias & GeoWiki & Grassland & 0.1 & 0.1 & 0.0 & 0.0 & 0.0 & 0.0 \\ 
  Agricultural & Bias & GlobCover & Grassland & -0.2 & -0.1 & -0.1 & -0.1 & -0.1 & -0.1 \\ 
  Agricultural & Bias & MODIS & Grassland & -0.3 & -0.2 & -0.2 & -0.1 & -0.1 & -0.1 \\ 
  Agricultural & Bias & SA-LC & Grassland & 0.3 & 0.2 & 0.2 & 0.1 & 0.1 & 0.1 \\ 
  Agricultural & Bias & GeoWiki & Sparse & -2.8 & -1.0 & -0.8 & -0.8 & -0.7 & -0.6 \\ 
  Agricultural & Bias & GlobCover & Sparse & -0.7 & 0.4 & 0.6 & 0.9 & 1.0 & 1.1 \\ 
  Agricultural & Bias & MODIS & Sparse & 2.9 & 2.9 & 2.8 & 2.7 & 2.6 & 2.5 \\ 
  Agricultural & Bias & SA-LC & Sparse & -6.7 & -3.8 & -3.3 & -2.9 & -2.8 & -2.5 \\ 
  Agricultural & MAE & GeoWiki & All & 26.8 & 9.4 & 6.5 & 4.5 & 3.4 & 2.4 \\ 
  Agricultural & MAE & GlobCover & All & 36.7 & 17.0 & 14.0 & 11.5 & 9.6 & 8.0 \\ 
  Agricultural & MAE & MODIS & All & 37.8 & 13.8 & 10.8 & 8.4 & 6.7 & 5.5 \\ 
  Agricultural & MAE & SA-LC & All & 13.6 & 6.3 & 4.9 & 3.9 & 3.4 & 3.0 \\ 
  Agricultural & MAE & GeoWiki & Forest & 45.9 & 14.3 & 9.8 & 6.6 & 4.8 & 3.4 \\ 
  Agricultural & MAE & GlobCover & Forest & 68.8 & 27.5 & 22.0 & 17.4 & 14.2 & 11.6 \\ 
  Agricultural & MAE & MODIS & Forest & 69.1 & 21.7 & 16.6 & 12.6 & 9.9 & 7.9 \\ 
  Agricultural & MAE & SA-LC & Forest & 22.1 & 9.6 & 7.5 & 5.8 & 4.9 & 4.3 \\ 
  Agricultural & MAE & GeoWiki & Secondary & 36.8 & 12.8 & 8.8 & 6.0 & 4.4 & 3.2 \\ 
  Agricultural & MAE & GlobCover & Secondary & 50.2 & 23.5 & 19.3 & 15.6 & 13.0 & 10.7 \\ 
  Agricultural & MAE & MODIS & Secondary & 52.3 & 19.1 & 14.9 & 11.4 & 9.1 & 7.3 \\ 
  Agricultural & MAE & SA-LC & Secondary & 18.7 & 8.6 & 6.7 & 5.3 & 4.5 & 4.0 \\ 
  Agricultural & MAE & GeoWiki & Shrubland & 35.8 & 12.6 & 8.7 & 6.0 & 4.4 & 3.2 \\ 
  Agricultural & MAE & GlobCover & Shrubland & 48.5 & 23.0 & 19.0 & 15.4 & 12.8 & 10.6 \\ 
  Agricultural & MAE & MODIS & Shrubland & 50.7 & 18.8 & 14.7 & 11.3 & 9.0 & 7.2 \\ 
  Agricultural & MAE & SA-LC & Shrubland & 18.3 & 8.4 & 6.6 & 5.2 & 4.5 & 3.9 \\ 
  Agricultural & MAE & GeoWiki & Grassland & 0.8 & 0.3 & 0.3 & 0.2 & 0.1 & 0.1 \\ 
  Agricultural & MAE & GlobCover & Grassland & 0.8 & 0.6 & 0.5 & 0.4 & 0.4 & 0.3 \\ 
  Agricultural & MAE & MODIS & Grassland & 0.9 & 0.5 & 0.4 & 0.3 & 0.3 & 0.2 \\ 
  Agricultural & MAE & SA-LC & Grassland & 0.5 & 0.2 & 0.2 & 0.2 & 0.1 & 0.1 \\ 
  Agricultural & MAE & GeoWiki & Sparse & 14.7 & 6.9 & 5.2 & 3.8 & 3.0 & 2.3 \\ 
  Agricultural & MAE & GlobCover & Sparse & 15.3 & 10.5 & 9.4 & 8.5 & 7.7 & 6.8 \\ 
  Agricultural & MAE & MODIS & Sparse & 16.2 & 9.0 & 7.6 & 6.3 & 5.4 & 4.6 \\ 
  Agricultural & MAE & SA-LC & Sparse & 8.6 & 4.4 & 3.7 & 3.1 & 2.9 & 2.6 \\ 
  Country & Bias & GeoWiki & All & -1.7 & 0.4 & 0.4 & 0.3 & 0.2 & 0.3 \\ 
  Country & Bias & GlobCover & All & -12.6 & -5.6 & -4.5 & -3.4 & -2.7 & -2.2 \\ 
  Country & Bias & MODIS & All & -7.3 & -2.8 & -2.3 & -1.9 & -1.7 & -1.5 \\ 
  Country & Bias & SA-LC & All & 2.1 & 2.0 & 1.8 & 1.7 & 1.5 & 1.5 \\ 
  Country & Bias & GeoWiki & Forest & -3.7 & 0.9 & 0.9 & 0.8 & 0.6 & 0.6 \\ 
  Country & Bias & GlobCover & Forest & -27.3 & -10.9 & -8.7 & -6.5 & -5.2 & -4.3 \\ 
  Country & Bias & MODIS & Forest & -15.8 & -5.7 & -4.6 & -4.0 & -3.6 & -3.2 \\ 
  Country & Bias & SA-LC & Forest & 4.5 & 4.2 & 3.9 & 3.6 & 3.3 & 3.2 \\ 
  Country & Bias & GeoWiki & Secondary & -2.1 & 0.8 & 0.8 & 0.7 & 0.5 & 0.6 \\ 
  Country & Bias & GlobCover & Secondary & -18.0 & -8.7 & -7.1 & -5.6 & -4.6 & -3.8 \\ 
  Country & Bias & MODIS & Secondary & -11.0 & -4.9 & -4.1 & -3.6 & -3.3 & -3.0 \\ 
  Country & Bias & SA-LC & Secondary & 3.9 & 3.7 & 3.5 & 3.3 & 3.1 & 3.0 \\ 
  Country & Bias & GeoWiki & Shrubland & -2.0 & 0.8 & 0.8 & 0.7 & 0.5 & 0.6 \\ 
  Country & Bias & GlobCover & Shrubland & -17.1 & -8.4 & -7.0 & -5.5 & -4.5 & -3.8 \\ 
  Country & Bias & MODIS & Shrubland & -10.6 & -4.8 & -4.1 & -3.5 & -3.2 & -2.9 \\ 
  Country & Bias & SA-LC & Shrubland & 3.8 & 3.6 & 3.5 & 3.2 & 3.0 & 2.9 \\ 
  Country & Bias & GeoWiki & Grassland & 0.0 & 0.0 & 0.0 & 0.0 & 0.0 & 0.0 \\ 
  Country & Bias & GlobCover & Grassland & -0.1 & -0.1 & -0.1 & -0.1 & -0.1 & -0.1 \\ 
  Country & Bias & MODIS & Grassland & -0.1 & -0.1 & -0.1 & -0.1 & -0.1 & -0.1 \\ 
  Country & Bias & SA-LC & Grassland & 0.1 & 0.1 & 0.1 & 0.1 & 0.1 & 0.1 \\ 
  Country & Bias & GeoWiki & Sparse & -0.9 & -0.5 & -0.5 & -0.5 & -0.5 & -0.5 \\ 
  Country & Bias & GlobCover & Sparse & -0.4 & 0.3 & 0.4 & 0.6 & 0.8 & 0.9 \\ 
  Country & Bias & MODIS & Sparse & 0.9 & 1.5 & 1.6 & 1.8 & 1.8 & 1.9 \\ 
  Country & Bias & SA-LC & Sparse & -1.9 & -1.9 & -1.9 & -1.9 & -1.9 & -1.9 \\ 
  Country & MAE & GeoWiki & All & 8.3 & 4.6 & 3.7 & 3.0 & 2.3 & 1.9 \\ 
  Country & MAE & GlobCover & All & 19.3 & 11.2 & 9.8 & 8.3 & 7.2 & 6.5 \\ 
  Country & MAE & MODIS & All & 12.5 & 7.2 & 6.4 & 5.5 & 4.7 & 4.1 \\ 
  Country & MAE & SA-LC & All & 3.9 & 3.0 & 2.8 & 2.5 & 2.4 & 2.3 \\ 
  Country & MAE & GeoWiki & Forest & 14.3 & 7.0 & 5.6 & 4.3 & 3.3 & 2.6 \\ 
  Country & MAE & GlobCover & Forest & 36.2 & 18.1 & 15.3 & 12.6 & 10.7 & 9.3 \\ 
  Country & MAE & MODIS & Forest & 22.9 & 11.4 & 9.8 & 8.2 & 6.9 & 5.9 \\ 
  Country & MAE & SA-LC & Forest & 6.3 & 4.7 & 4.2 & 3.8 & 3.4 & 3.3 \\ 
  Country & MAE & GeoWiki & Secondary & 11.5 & 6.3 & 5.0 & 3.9 & 3.1 & 2.4 \\ 
  Country & MAE & GlobCover & Secondary & 26.4 & 15.5 & 13.4 & 11.3 & 9.8 & 8.6 \\ 
  Country & MAE & MODIS & Secondary & 17.3 & 10.0 & 8.7 & 7.4 & 6.3 & 5.5 \\ 
  Country & MAE & SA-LC & Secondary & 5.3 & 4.2 & 3.8 & 3.4 & 3.2 & 3.0 \\ 
  Country & MAE & GeoWiki & Shrubland & 11.2 & 6.2 & 5.0 & 3.9 & 3.0 & 2.4 \\ 
  Country & MAE & GlobCover & Shrubland & 25.5 & 15.2 & 13.2 & 11.2 & 9.7 & 8.5 \\ 
  Country & MAE & MODIS & Shrubland & 16.8 & 9.8 & 8.6 & 7.3 & 6.2 & 5.4 \\ 
  Country & MAE & SA-LC & Shrubland & 5.2 & 4.1 & 3.8 & 3.4 & 3.1 & 3.0 \\ 
  Country & MAE & GeoWiki & Grassland & 0.2 & 0.2 & 0.1 & 0.1 & 0.1 & 0.1 \\ 
  Country & MAE & GlobCover & Grassland & 0.4 & 0.4 & 0.3 & 0.3 & 0.3 & 0.3 \\ 
  Country & MAE & MODIS & Grassland & 0.3 & 0.2 & 0.2 & 0.2 & 0.2 & 0.2 \\ 
  Country & MAE & SA-LC & Grassland & 0.1 & 0.1 & 0.1 & 0.1 & 0.1 & 0.1 \\ 
  Country & MAE & GeoWiki & Sparse & 4.6 & 3.4 & 3.0 & 2.5 & 2.1 & 1.7 \\ 
  Country & MAE & GlobCover & Sparse & 8.0 & 6.9 & 6.6 & 6.2 & 5.8 & 5.5 \\ 
  Country & MAE & MODIS & Sparse & 5.4 & 4.7 & 4.5 & 4.1 & 3.8 & 3.5 \\ 
  Country & MAE & SA-LC & Sparse & 2.5 & 2.2 & 2.1 & 2.0 & 2.0 & 2.0 \\ 
  Density & Bias & SA-LC & All & 10.9 & 9.6 & 8.2 & 6.5 & 5.0 & 4.2 \\ 
  Density & Bias & GlobCover & All & -123.4 & -47.6 & -35.9 & -24.8 & -17.4 & -12.3 \\ 
  Density & Bias & MODIS & All & -66.0 & -17.6 & -12.0 & -8.3 & -6.2 & -4.1 \\ 
  Density & Bias & GeoWiki & All & -20.4 & 2.1 & 2.3 & 1.3 & 0.3 & 0.5 \\ 
  Density & Bias & SA-LC & Forest & 22.7 & 19.7 & 16.9 & 13.3 & 10.4 & 9.0 \\ 
  Density & Bias & GlobCover & Forest & -276.2 & -98.3 & -73.3 & -50.2 & -35.5 & -25.4 \\ 
  Density & Bias & MODIS & Forest & -146.5 & -36.1 & -24.5 & -17.0 & -12.9 & -8.8 \\ 
  Density & Bias & GeoWiki & Forest & -46.1 & 4.3 & 4.6 & 2.7 & 0.6 & 1.0 \\ 
  Density & Bias & SA-LC & Secondary & 18.4 & 16.7 & 14.6 & 11.8 & 9.5 & 8.2 \\ 
  Density & Bias & GlobCover & Secondary & -186.3 & -79.3 & -61.2 & -43.8 & -31.7 & -23.2 \\ 
  Density & Bias & MODIS & Secondary & -101.0 & -30.6 & -21.5 & -15.2 & -11.7 & -8.0 \\ 
  Density & Bias & GeoWiki & Secondary & -30.5 & 3.4 & 3.7 & 2.2 & 0.6 & 0.9 \\ 
  Density & Bias & SA-LC & Shrubland & 17.9 & 16.4 & 14.3 & 11.6 & 9.4 & 8.1 \\ 
  Density & Bias & GlobCover & Shrubland & -178.2 & -77.1 & -59.8 & -42.9 & -31.2 & -22.9 \\ 
  Density & Bias & MODIS & Shrubland & -96.8 & -29.9 & -21.1 & -15.0 & -11.5 & -7.9 \\ 
  Density & Bias & GeoWiki & Shrubland & -29.2 & 3.3 & 3.6 & 2.2 & 0.6 & 0.9 \\ 
  Density & Bias & SA-LC & Grassland & 0.3 & 0.3 & 0.3 & 0.3 & 0.2 & 0.2 \\ 
  Density & Bias & GlobCover & Grassland & -1.9 & -1.2 & -1.1 & -0.9 & -0.8 & -0.6 \\ 
  Density & Bias & MODIS & Grassland & -1.1 & -0.6 & -0.5 & -0.4 & -0.3 & -0.2 \\ 
  Density & Bias & GeoWiki & Grassland & -0.3 & 0.0 & 0.1 & 0.0 & 0.0 & 0.0 \\ 
  Density & Bias & SA-LC & Sparse & -4.6 & -5.2 & -5.1 & -4.8 & -4.6 & -4.4 \\ 
  Density & Bias & GlobCover & Sparse & 25.4 & 18.1 & 16.1 & 13.9 & 12.2 & 10.5 \\ 
  Density & Bias & MODIS & Sparse & 15.4 & 9.1 & 7.6 & 6.3 & 5.5 & 4.4 \\ 
  Density & Bias & GeoWiki & Sparse & 4.0 & -0.3 & -0.6 & -0.5 & -0.3 & -0.4 \\ 
  Density & MAE & SA-LC & All & 19.2 & 12.5 & 10.7 & 8.6 & 6.9 & 6.0 \\ 
  Density & MAE & GlobCover & All & 134.9 & 56.2 & 43.8 & 31.9 & 23.9 & 18.2 \\ 
  Density & MAE & MODIS & All & 84.8 & 33.2 & 26.2 & 19.9 & 14.9 & 11.4 \\ 
  Density & MAE & GeoWiki & All & 47.3 & 17.9 & 12.8 & 8.8 & 5.8 & 3.9 \\ 
  Density & MAE & SA-LC & Forest & 34.8 & 21.0 & 17.5 & 13.6 & 10.6 & 9.1 \\ 
  Density & MAE & GlobCover & Forest & 278.2 & 100.3 & 75.3 & 52.4 & 37.6 & 27.7 \\ 
  Density & MAE & MODIS & Forest & 168.6 & 56.2 & 42.9 & 31.6 & 22.9 & 17.1 \\ 
  Density & MAE & GeoWiki & Forest & 90.5 & 29.9 & 20.9 & 14.0 & 8.9 & 5.8 \\ 
  Density & MAE & SA-LC & Secondary & 27.4 & 17.9 & 15.2 & 12.1 & 9.6 & 8.3 \\ 
  Density & MAE & GlobCover & Secondary & 188.1 & 81.1 & 63.1 & 45.7 & 33.8 & 25.3 \\ 
  Density & MAE & MODIS & Secondary & 118.9 & 47.6 & 37.3 & 28.1 & 20.8 & 15.7 \\ 
  Density & MAE & GeoWiki & Secondary & 66.6 & 25.5 & 18.1 & 12.4 & 8.0 & 5.4 \\ 
  Density & MAE & SA-LC & Shrubland & 26.6 & 17.6 & 14.9 & 11.9 & 9.5 & 8.2 \\ 
  Density & MAE & GlobCover & Shrubland & 179.9 & 79.0 & 61.7 & 44.9 & 33.2 & 24.9 \\ 
  Density & MAE & MODIS & Shrubland & 114.2 & 46.6 & 36.6 & 27.7 & 20.5 & 15.5 \\ 
  Density & MAE & GeoWiki & Shrubland & 64.3 & 24.9 & 17.8 & 12.2 & 7.9 & 5.3 \\ 
  Density & MAE & SA-LC & Grassland & 0.4 & 0.4 & 0.3 & 0.3 & 0.2 & 0.2 \\ 
  Density & MAE & GlobCover & Grassland & 1.9 & 1.3 & 1.1 & 1.0 & 0.8 & 0.7 \\ 
  Density & MAE & MODIS & Grassland & 1.4 & 0.9 & 0.8 & 0.7 & 0.6 & 0.5 \\ 
  Density & MAE & GeoWiki & Grassland & 0.9 & 0.5 & 0.4 & 0.3 & 0.2 & 0.2 \\ 
  Density & MAE & SA-LC & Sparse & 6.7 & 5.8 & 5.4 & 4.9 & 4.7 & 4.4 \\ 
  Density & MAE & GlobCover & Sparse & 26.4 & 19.6 & 17.7 & 15.7 & 14.0 & 12.3 \\ 
  Density & MAE & MODIS & Sparse & 20.7 & 14.9 & 13.3 & 11.5 & 9.7 & 8.3 \\ 
  Density & MAE & GeoWiki & Sparse & 14.1 & 8.4 & 6.6 & 5.1 & 3.9 & 2.9 \\ 
   \hline
\hline
\end{longtable}



\section*{Yield and Harvested Area Bias}
\begin{figure}[ht]
  \centering
     \includegraphics[width = 12cm]{figures/cropland_adj_bias_map2.pdf} 
      \caption{Errors in cropland maps adjusted using provincial cropland area statistics.}
      \label{fig:default}
\end{figure}

\FloatBarrier
% latex table generated in R 3.4.1 by xtable 1.8-2 package
% Wed Aug 30 17:02:57 2017
\begin{longtable}{llrrrrrr}
\caption{Bias and mean absolute errors (MAE) in statistically constrained cropland maps across aggregation scales, weighted by density of cropland cover in the reference map. } \\ 
  \hline
Metric & Map & 1 km & 5 km & 10 km & 25 km & 50 km & 100 km \\ 
  \hline
Bias & GLC-Share & 9.7 & 1.1 & 0.6 & 0.4 & 0.5 & 0.1 \\ 
  Bias & GlobCover & 34.5 & 18.3 & 14.5 & 10.6 & 7.6 & 4.6 \\ 
  Bias & MODIS & 17.8 & 5.5 & 3.2 & 1.3 & 0.1 & -1.3 \\ 
  Bias & SA-LC & 6.6 & 2.7 & 2.1 & 1.6 & 1.1 & 0.6 \\ 
  Accuracy & GLC-Share & 23.8 & 12.6 & 9.4 & 6.8 & 4.8 & 3.0 \\ 
  Accuracy & GlobCover & 42.3 & 27.3 & 23.3 & 18.8 & 15.6 & 11.2 \\ 
  Accuracy & MODIS & 33.8 & 21.5 & 18.4 & 15.3 & 12.7 & 10.6 \\ 
  Accuracy & SA-LC & 11.4 & 6.0 & 4.7 & 3.7 & 2.8 & 1.9 \\ 
   \hline
\hline
\end{longtable}


\clearpage
\begin{figure}[ht]
  \centering
     \includegraphics[width = 12cm]{figures/yld_bias_map.pdf} 
      \caption{Errors (normalized to the reference-derived country mean) in disaggregated maize yield estimates.}
      \label{fig:default}
\end{figure}

\begin{figure}[ht]
  \centering
     \includegraphics[width = 12cm]{figures/prod_bias_map.pdf} 
      \caption{Errors (normalized to reference-derived country mean) production estimates calculated from disaggregated maize yield and harvested area estimates.}
      \label{fig:default}
\end{figure}

\FloatBarrier
% latex table generated in R 3.2.1 by xtable 1.7-4 package
% Thu Sep 24 07:06:08 2015
\begin{longtable}{lllrrrrrr}
\caption{Biases and mean absolute errors (MAE) in disaggregated maize yield and production (calculated from disaggregated yield and harvested area estimates) maps. Bias and MAE were normalized to their respective mean values calculated from reference maps.} \\ 
  \hline
Metric & Map & Variable & 1 km & 5 km & 10 km & 25 km & 50 km & 100 km \\ 
  \hline
Bias & SA-LC & Yield & -5.1 & -0.3 & 2.9 & 3.5 & 3.5 & 1.5 \\ 
  Bias & GlobCover & Yield & -58.0 & -35.5 & -21.9 & -11.7 & -8.8 & -1.5 \\ 
  Bias & MODIS & Yield & 5.1 & 21.1 & 28.7 & 26.4 & 20.2 & 11.4 \\ 
  Bias & GeoWiki & Yield & 2.4 & 24.1 & 29.0 & 24.9 & 21.1 & 9.6 \\ 
  Bias & SA-LC & Production & 0.0 & -0.1 & -0.1 & -0.1 & -0.0 & 0.0 \\ 
  Bias & GlobCover & Production & 0.0 & -0.1 & 0.0 & 0.1 & 0.3 & 0.3 \\ 
  Bias & MODIS & Production & 0.0 & -0.1 & -0.1 & -0.1 & 0.0 & -0.1 \\ 
  Bias & GeoWiki & Production & 0.0 & 0.1 & 0.0 & 0.0 & 0.1 & 0.1 \\ 
  MAE & SA-LC & Yield & 15.5 & 16.4 & 19.6 & 15.5 & 12.0 & 6.7 \\ 
  MAE & GlobCover & Yield & 71.7 & 47.6 & 37.4 & 23.1 & 17.0 & 6.1 \\ 
  MAE & MODIS & Yield & 55.9 & 50.5 & 50.0 & 44.2 & 37.8 & 20.4 \\ 
  MAE & GeoWiki & Yield & 41.1 & 40.5 & 39.8 & 34.5 & 28.1 & 14.3 \\ 
  MAE & SA-LC & Production & 19.7 & 11.3 & 8.6 & 5.5 & 3.3 & 1.9 \\ 
  MAE & GlobCover & Production & 55.7 & 55.5 & 52.5 & 42.2 & 28.1 & 17.3 \\ 
  MAE & MODIS & Production & 56.0 & 41.3 & 35.6 & 24.9 & 14.1 & 8.4 \\ 
  MAE & GeoWiki & Production & 43.7 & 30.2 & 23.5 & 15.3 & 9.3 & 4.0 \\ 
   \hline
\hline
\end{longtable}


\section*{Evapotranspiration bias}
%% latex table generated in R 3.2.1 by xtable 1.7-4 package
% Wed Aug 19 16:18:01 2015
\begin{table}[ht]
\centering
\begin{tabular}{rllrr}
  \hline
 & ETvar & dataset & bias & absbias \\ 
  \hline
1 & gs & geow & 0.2 & 0.8 \\ 
  2 & gspk & geow & 0.2 & 0.8 \\ 
  3 & mu & geow & 0.3 & 0.7 \\ 
  4 & mumn & geow & 0.3 & 0.6 \\ 
  5 & mumx & geow & 0.3 & 0.8 \\ 
  6 & gs & globmu & 0.1 & 1.2 \\ 
  7 & gspk & globmu & 0.2 & 1.4 \\ 
  8 & mu & globmu & -0.1 & 1.0 \\ 
  9 & mumn & globmu & -0.2 & 0.9 \\ 
  10 & mumx & globmu & 0.2 & 1.2 \\ 
  11 & gs & modmu & -0.5 & 0.9 \\ 
  12 & gspk & modmu & -0.4 & 1.0 \\ 
  13 & mu & modmu & -0.6 & 0.8 \\ 
  14 & mumn & modmu & -0.6 & 0.7 \\ 
  15 & mumx & modmu & -0.5 & 0.8 \\ 
  16 & gs & sa30 & 0.3 & 0.7 \\ 
  17 & gspk & sa30 & 0.2 & 0.8 \\ 
  18 & mu & sa30 & 0.5 & 0.6 \\ 
  19 & mumn & sa30 & 0.4 & 0.5 \\ 
  20 & mumx & sa30 & 0.4 & 0.6 \\ 
   \hline
\end{tabular}
\end{table}


\section*{Agent allocation bias}

\begin{figure}[ht]
  \centering
     \includegraphics[width = 12cm]{figures/abm-selected-districts.pdf} 
      \caption{The location of the four selected magisterial districts (top left) used in evaluating agent allocation bias, the reference levels of cropland cover within those districts (top right), and the difference in cropland percentage between the reference and each of the four cropland maps (lower four panels). }
      \label{fig:default}
\end{figure}



\FloatBarrier
%\section*{References}
\bibliographystyle{pnas2011} 
{\footnotesize \bibliography{/Users/lestes/Dropbox/publications/full.bib}}
\end{document}  

